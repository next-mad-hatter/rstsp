%
% $Id$
% $Author$
% $Date$
% $Revision$
%

\documentclass[titlepage]{scrartcl}
\KOMAoptions{fontsize=12pt}

\usepackage{geometry}
\geometry{top=2.2cm,bottom=1cm}

\usepackage{fontspec}
\defaultfontfeatures{Ligatures=TeX}

\setmainfont[Numbers=OldStyle]{Linux Libertine O}
\setsansfont[Numbers=OldStyle]{Linux Biolinum O}
\setmonofont[Scale=0.87]{Linux Libertine Mono O}

\usepackage{luatextra}

\usepackage[english,ngerman,russian]{babel}

\usepackage{graphicx}

\usepackage{xcolor}
\definecolor{blue}{rgb}{0,0,0.3}
%\definecolor{brown}{HTML}{796045}
%\definecolor{brown2}{HTML}{BFAC60}
%\definecolor{green}{HTML}{AEBF60}
%\definecolor{yellow}{HTML}{FFF3BF}

\usepackage{siunitx}

\usepackage[pdfauthor={M. Deineko},
            pdfproducer={lualatex},
            pdfcreator={LaTeX},
            colorlinks=true,
            linkcolor=blue,
            urlcolor=blue,
            citecolor=blue,
            linktoc=section,
            unicode=true
            ]{hyperref}
\hypersetup{pdftitle=Two Exponential Neighbourhoods}
\usepackage[nameinlink]{cleveref}

\let\defstyle\itshape
%\def\defstyle{\itshape}

\usepackage{pgfornament}

%\usepackage{eso-pic}
%\usepackage{calc}
%
%\makeatletter
%\AddToShipoutPicture{%
%\begingroup
%\setlength{\@tempdima}{4mm}%
%\setlength{\@tempdimb}{\paperwidth-\@tempdima-1.9cm}%
%\setlength{\@tempdimc}{\paperheight-\@tempdima}%
%\put(\LenToUnit{\@tempdima},\LenToUnit{\@tempdimc}){%
%\pgfornament[anchor=north west,width=1.9cm,symmetry=h]{79}}
%\put(\LenToUnit{\@tempdima},\LenToUnit{\@tempdima}){%
%\pgfornament[anchor=south west,width=1.9cm,symmetry=h]{77}}
%\put(\LenToUnit{\@tempdimb},\LenToUnit{\@tempdimc}){%
%\pgfornament[anchor=north east,width=1.9cm,symmetry=v]{77}}
%\put(\LenToUnit{\@tempdimb},\LenToUnit{\@tempdima}){%
%\pgfornament[anchor=south east,width=1.9cm,symmetry=v]{79}}
%\endgroup
%}
%\makeatother

\begin{document}

\thispagestyle{empty}
%\pagecolor{yellow}

\selectlanguage{english}

\hfill\includegraphics[width=36mm]{./tug.png}

\begin{center}
\vspace{2.4cm}

{\sffamily {\bfseries {\scshape {\Large
  Two Exponential Neighbourhoods for the TSP
  and Related Heuristics
}}}}
\\
\vspace{0.52cm}
{\sffamily {\itshape {\large Pyramidal and strongly balanced tours -- theory and implementation}}}

\vspace{1.65cm}
\pgfornament[anchor=center,scale=0.31]{87}
\vspace{1.6cm}

{\bfseries Bachelorarbeit}

\vspace{1.6cm}
Eingereicht für das
Bachelorstudium
\\
Technische Mathematik
\\
der Technischen Universität Graz

\vspace{0.8cm}
Vorgelegt von
\\
\href{mailto:deineko@student.tugraz.at}{Maksym Deineko}

\vspace{0.8cm}
Betreut durch
\\
Ao.Univ.-Prof.\ Dipl.-Ing.\ Dr.techn.
\href{mailto:cela@math.tugraz.at}{Eranda Dragoti-Cela}

\vspace{1.7cm}
September 2016

\end{center}

\end{document}

