%
% $Author$
% $File$
% $Date$
% $Revision$
%

\documentclass[
  size=10pt,
%%  style=paintings,
  style=klope,
%%  style=jefka,
%%  style=aggie,
  paper=screen,
%%  orient=portrait,
%%  mode=handout,
%%  display=slidesnotes,
  pauseslide,
  nopagebreaks,
  fleqn
]{powerdot}

%% paintings palettes: Syndics (the default), Skater , GoldenGate , Lamentation , HolyWood , Europa , Moitessier , MayThird , PearlEarring and Charon
%% klope palettes: Spring , PastelFlower , BlueWater and BlackWhite
%% jefka palettes: brown (the default), seagreen , blue and white
%\pdsetup{palette=BlackWhite}
%\pdsetup{palette=white}

\pdsetup{
  itemize={labelsep=0.2cm},
  lf=two exponential neighbourhoods,
%%  rf=why not,
  trans=Wipe,
  theslide=slide~\arabic{slide},
  list={itemsep=6pt}
}

\usepackage[utf8]{inputenc}
\usepackage[english]{babel}

%\usepackage[T1]{fontenc}
%\usepackage[T2A]{fontenc}
%\usepackage{microtype}

%\usepackage{fix-cm}
%\usepackage{lmodern}
%\usepackage{mathdesign}
\usepackage{amsmath}
\usepackage{amsfonts}
\usepackage{amssymb}

\usepackage{commath}
\usepackage{mathtools}
\usepackage{amsthm}

\usepackage{graphicx}
\usepackage{wrapfig}

\DeclareMathOperator*{\argmin}{arg\,min}
\let\defstyle\itshape

\title{Two Exponential Neighbourhoods}
\newcommand{\mailto}[1]{\href{mailto:#1}{\nolinkurl{#1}}}
\author{Maksym~Deineko \\ {\small \mailto{max.deineko@gmail.com} }}

\begin{document}

\maketitle

\section[template=wideslide,tocsection=false,slide=false]{Overview}

\begin{slide}[toc=,bm=]{Overview}
\tableofcontents[content=all,type=1]
\end{slide}

\section[template=wideslide]{Introduction}

\begin{slide}{Paths and Cities}
\begin{itemize}
  \item
  Cities:
  \begin{equation}
    %\text{Cities:} \quad
    \mathcal{N} \coloneqq \mathcal{N}_n \coloneqq \left\{1,\ldots,n\right\}
    \quad (n \in \mathbb{N}).
  \end{equation}%\pause
  \item
  Paths:
  \begin{equation}
    %\text{Paths:} \quad
    \left(p_1,p_2,\ldots,p_m\right) \in \bigcup_{k \in \mathbb{N}}\mathcal{N}^k
    \eqqcolon \mathcal{P}_n %\supseteq \mathcal{T}_n.
    %\quad \text{(length, end points, simple, open, closed)}
  \end{equation}%\pause
   (length, end points, simple, open, closed).
  \item
  Path concatenation:
  \begin{equation}
    \oplus: {\left( \mathcal{P}_n \cup 2^{\mathcal{P}_n} \right)}^2 \to \mathcal{P}_n \cup 2^{\mathcal{P}_n}
  \end{equation}%\pause
  (for paths: common definition, for sets: element-wise).
\end{itemize}
\end{slide}

\begin{slide}{Costs and Tours}
\begin{itemize}
  \item
  Cost matrix:
  \begin{equation}
    C = \left(c_{ij}\right) \in~\mathbb{R}^{n \times n},
    \quad n \in \mathbb{N} \left(n \geq 2\right).
  \end{equation}
  \item
  Path cost:
  \begin{equation}
    \mathfrak{w}_C: \bigcup_{k \in \mathbb{N}, k \geq 2}\mathcal{N}^k \to \mathbb{R}, \quad
    \left(p_1,p_2,\ldots,p_m\right) \mapsto \sum_{k=1}^{m-1} c_{p_k p_{k+1}}.
  \end{equation}
  \item
  Tours: $\mathcal{P}_n \supset \mathcal{T}_n \sim \mathcal{S}_n$ (fixed starting city).
  \item
  Associated permutation:
  \begin{equation}
  \tau = \left(p_1, p_2, \ldots, p_n, p_1 \right)
  \; \sim \;
  \mathfrak{s}_\tau \coloneqq
  \begin{pmatrix}
    1 & 2 & \cdots & n \\
    p_1 & p_2 & \cdots &  p_n
  \end{pmatrix}.
  \end{equation}
\end{itemize}
\end{slide}

\begin{slide}{Traveling Salesman Problem}
\begin{itemize}
  \item
  Traveling Salesman Problem:
  \begin{equation}
    \operatorname{TSP}_{M,T}: M \to T, \quad
    C \mapsto \argmin_{\tau \in T} \mathfrak{w}_C\left(\tau\right)
    \qquad \text{for} \; M \subseteq~\mathbb{R}^{n \times n}, T \subseteq \mathcal{T}_n.
  \end{equation}
  \item
  NP-hard.
  \item
  Variations:
  \begin{itemize}
  \item return cost along with (instead of) tour;
  \item $M, T$ as maps from $\mathbb{N}$;
  \item (s)TSP vs (a)TSP;
  \item $c_{ij} \geq 0$;
  \item $\mathcal{S}_n$ vs $\mathcal{T}_n$.
  \end{itemize}
  \item
  Classes/cases: restrictions to subsets of $(\mathbb{R}^{n \times n}, \mathcal{T}_n)$.
\end{itemize}
\end{slide}

\begin{slide}{Exponential Neighbourhoods}
\begin{itemize}
  \item
  Neighbourhood:
  \begin{equation}
    F: \mathcal{T}_n \to 2^{\mathcal{T}_n};
  \end{equation}
  exponential neighbourhood:
  $\left\vert F(\tau) \right\vert = \Omega(2^n) \qquad \forall \tau \in \mathcal{T}_n$.
  %\end{equation}
  \item
  Local Search:
  \begin{equation}
    l: \tau \mapsto \operatorname{TSP}_{M,F(\tau)};
    \qquad \text{(sometimes: anytime heuristic)}.
  \end{equation}
  \item
  Iterative Local Search:
  \begin{align}
    \text{compute fixed point of}
    \quad
    \tau \mapsto
    \begin{cases}
      \tau \quad & \text{if} \; \mathfrak{w}\left(l\left(\tau\right)\right) = \mathfrak{w}\left(\tau\right),
      \\
      l\left(\tau\right) & \text{otherwise};
    \end{cases}
  \end{align}
  termination conditions may vary.
\end{itemize}
\end{slide}

\begin{slide}[toc=,bm=]{Overview}
\tableofcontents[content=currentsection,type=1]
\end{slide}

\section[template=wideslide]{Theory}

\begin{slide}{Motivation: Pyramidal Tours}
\begin{itemize}
  \item
  Pyramidal path:
  \begin{align}
  \left( p_1, \ldots, p_k, q_1, \ldots, q_m \right) \quad \text{s.t.} \quad
  \begin{cases}
  k+m \geq 1, & \\
  p_i < p_{i+1} \; & \forall i \in \mathcal{N}_{k-1},\\
  q_j > q_{j+1} \; & \forall j \in \mathcal{N}_{m-1}.
  \end{cases}
  \end{align}
  \item
  Neighbourhood $\operatorname{Pyr}\left(\tau\right):$
  tours associated with of $\mathfrak{s}_\pi \circ \mathfrak{s}_\tau$
  for all pyramidal tours $\pi$.
  \begin{figure}[H]
    \centering
  %\begin{wrapfigure}{r}{0.5\textwidth}
  %\begin{center}
    %\includegraphics[width=0.49\textwidth]{../plot/build/pyr.1}
    \includegraphics[width=0.54\textwidth]{../plot/build/pyr.2}
    \caption{%
      Connected plot of $\mathfrak{s}_\pi$'s graph resembles a pyramid
      for a pyramidal $\pi$.
    }
  %\end{center}
  %\end{wrapfigure}
  \end{figure}
\end{itemize}
\end{slide}

\begin{slide}{Pyramidal Tours: Recurse!}
%  \begin{figure}[H]
%    \centering
%    \includegraphics[width=0.34\textwidth]{../plot/build/pyr.2}
%    \caption{TODO: elaborate.}
%  \end{figure}
\begin{itemize}
  \item
  %All pyramidal tours:
  %\begin{equation}
  %  42.
  %\end{equation}
  %\item
  Dynamic programming (s.\ blackboard):
  \begin{align}
    \Phi_C\left(i,j\right) \coloneqq \; & \text{minimum cost pyramidal path} \left(i,p_1,\ldots,p_m,j\right) \\
    \text{s.t.} \; & \left\{ p_1,\ldots,p_m \right\} = \left\{ k,\ldots,n \right\}
    \text{with} \; k = \max\left\{i,j\right\}+1.
  \end{align}
  \item
  It then follows:
    \begin{align}
    \Phi_C\left(i,n\right) \; = \; & \left(i,n\right), \quad \forall i \in \mathcal{N};
    \\
    \Phi_C\left(n,j\right) \; = \; & \left(n,j\right) \quad \forall j \in \mathcal{N};
    \\
    \Phi_C\left(i,j\right) \; = \; & \argmin_{\tau \in \left\{\tau\prime, \tau\prime\prime\right\}}
    \mathfrak{w}_C\left(\tau\right),
    \\
    \tau\prime \; = \; & \left(i\right) \oplus \Phi\left(k,j\right),
    \\
    \tau\prime\prime \; = \; & \Phi\left(i,k\right) \oplus \left(j\right),
    \\
    \text{for} \; & i,j < n, \quad k = \max\left\{i,j\right\}+1;
    \\
    \Phi_C\left(1,1\right) \; = \; &
    \operatorname{TSP}_{\left\{C\right\},\operatorname{Pyr\left(1,2,\ldots,n,1\right)}}\left(C\right).
    \end{align}
\end{itemize}
\end{slide}

\begin{slide}{Pyramidal Tours and Quadratic Time}
  \begin{figure}[H]
    \centering
    \includegraphics[width=0.20\textwidth]{../plot/build/trace_apyr.eps}
    \caption{Pyramidal tours: recursion graph.}
  \end{figure}
  \begin{itemize}
  \item
  Dynamic programming solution yields depth first search.
  \item
  Quadratic time assuming:
  \begin{itemize}
    \item memoization: add subset of $\Phi$'s graph to its arguments;
    \item constant time memory access (practice: good hash not always obvious);
    \item constant time tour construction;
    \item constant time tour comparison (compute cost in $\Phi$).
  \end{itemize}
  \item
  BFS: linear space overhead.
  \begin{equation}
  \end{equation}
  \end{itemize}
\end{slide}

\begin{slide}[toc=,bm=]{Overview}
\tableofcontents[content=currentsection,type=1]
\end{slide}

\section[template=wideslide]{Implementation}

\begin{slide}[toc=,bm=]{Overview}
\tableofcontents[content=currentsection,type=1]
\end{slide}

\section[template=wideslide]{Conclusions}

\begin{slide}{Template}
  \begin{itemize}
  \item
  Something
  \begin{equation}
  \end{equation}
  \end{itemize}
\end{slide}

\begin{note}{Note}
  Some note.
\end{note}

\begin{wideslide}{Yes}
  \begin{center}
%%    \includegraphics[width=0.80\textwidth]{./test.eps}
%%    \includegraphics[width=0.50\textwidth]{../plot/build/trace_apyr.eps}
    \includegraphics[width=0.69\textwidth]{../plot/build/tsplib_time.ps}
  \end{center}
\end{wideslide}

\end{document}
\endinput
