%
% $Author$
% $File$
% $Date$
% $Revision$
%

\documentclass[
  size=12pt,
  style=paintings,
%%  style=klope,
%%  style=jefka,
%%  style=aggie,
  paper=screen,
%%  orient=portrait,
%%  mode=handout,
%%  display=slidesnotes,
  pauseslide,
  nopagebreaks,
  fleqn
]{powerdot}

%% paintings palettes: Syndics (the default), Skater , GoldenGate , Lamentation , HolyWood , Europa , Moitessier , MayThird , PearlEarring and Charon
%% klope palettes: Spring , PastelFlower , BlueWater and BlackWhite
%% jefka palettes: brown (the default), seagreen , blue and white
%\pdsetup{palette=BlackWhite}
%\pdsetup{palette=white}

\pdsetup{
  lf=two exponential neighbourhoods,
  rf=why not,
  trans=Wipe,
  theslide=slide~\arabic{slide},
  list={itemsep=6pt}
}

\title{Two Exponential Neighbourhoods}
\newcommand{\mailto}[1]{\href{mailto:#1}{\nolinkurl{#1}}}
\author{Maksym~Deineko \\ {\small \mailto{max.deineko@gmail.com} }}

\usepackage[utf8]{inputenc}
\usepackage[ngerman]{babel}

%\usepackage[T1]{fontenc}
%\usepackage[T2A]{fontenc}
%\usepackage{microtype}

%\usepackage{fix-cm}
%\usepackage{lmodern}

%\usepackage{amsfonts}
%\usepackage{amssymb}
%\usepackage{amsmath}

%\usepackage{mathdesign}

\usepackage{amsthm}

\begin{document}

\maketitle

\begin{slide}{Overview}
  \begin{itemize}[type=1]
    \item One\pause
    \item Two.
  \end{itemize}
\end{slide}

\section{Introduction}

\begin{slide}{TSP}
  \begin{equation}\label{binomium}
    (a+b)^n=\sum_{k=0}^n{n\choose k}a^{n-k}b^k
  \end{equation}\pause
  We will prove formula (\ref{binomium}) on the blackboard.\\
\end{slide}

\begin{note}{Note}
  Some note.
\end{note}

\section[template=wideslide,tocsection=hidden]{Hidden section}

\begin{wideslide}{Yes}
  \begin{center}
    \includegraphics[width=0.80\textwidth]{./test.eps}
  \end{center}
\end{wideslide}

\end{document}
\endinput
