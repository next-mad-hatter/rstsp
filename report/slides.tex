%;
% $Author$
% $File$
% $Date$
% $Revision$
%

\documentclass[
  size=8pt,
  style=klope,
%%  style=jefka,
%%  style=paintings,
  paper=screen,
%%  orient=portrait,
  mode=present,
%%  mode=handout,
%%  display=slidesnotes,
  nohandoutpagebreaks,
  pauseslide,
  hlsections,
  fleqn,
%%  clock
]{powerdot}

%% paintings palettes: Syndics (the default), Skater , GoldenGate , Lamentation , HolyWood , Europa , Moitessier , MayThird , PearlEarring and Charon
%% klope palettes: Spring , PastelFlower , BlueWater and BlackWhite
%% jefka palettes: brown (the default), seagreen , blue and white
%\pdsetup{palette=BlackWhite}
%\pdsetup{palette=white}

\pdsetup{%
  itemize={labelsep=0.2cm},
  lf=two exponential neighbourhoods,
%%  rf=why not,
  trans=Wipe,
  theslide=slide~\arabic{slide},
  list={itemsep=6pt}
}

\usepackage[utf8]{inputenc}
\usepackage[english]{babel}

%\usepackage[T1]{fontenc}
%\usepackage[T2A]{fontenc}
%\usepackage{microtype}

%\usepackage{fix-cm}
%\usepackage{lmodern}
%\usepackage{mathdesign}
\usepackage{amsmath}
\usepackage{amsfonts}
\usepackage{amssymb}

\usepackage{commath}
\usepackage{mathtools}
\usepackage{amsthm}
\usepackage{thmtools}
\declaretheorem[style=definition,name=Definition,thmbox=L]{define}
\declaretheorem[style=plain]{theorem}
\declaretheorem[sibling=theorem]{lemma}
\declaretheorem[style=remark]{remark}
%% FIXME: framing possible with thmtools?
%\usepackage[framemethod=pstricks]{mdframed}
%\surroundwithmdframed[outerlinewidth=2,roundcorner=10pt]{define}
%\usepackage{pstricks}
%\usepackage{fancybox}
%\usepackage{fancyvrb}

%\usepackage{minted}
%\usepackage{verbments}

\usepackage{graphicx}
%% possible with powerdot?
%\usepackage{wrapfig}
%\usepackage{floatflt}
\usepackage{xcolor}
\usepackage{siunitx}
\usepackage{booktabs}

\DeclareMathOperator*{\argmin}{arg\,min}
%\newcommand*{\Cdot}{\raisebox{-0.25ex}{\scalebox{1.2}{$\cdot$}}}
\newcommand*{\Cdot}{\raisebox{-0.30ex}{\scalebox{1.5}{$\cdot$}}}
\let\defstyle\itshape%
\definecolor{red}{rgb}{0.6,0.1,0}
\definecolor{green}{rgb}{0.68,0.74,0.38}
\def\todo{\color{red}}
\def\board{{\color{green} [Optional: board.]}}
% FIXME: automate?
\def\eqitspace{\vspace{-5mm}}

\title{Two Exponential Neighbourhoods for the TSP and related heuristics}
\newcommand{\mailto}[1]{\href{mailto:#1}{\nolinkurl{#1}}}
\author{Maksym~Deineko \\ {\small \mailto{max.deineko@gmail.com} }}

\begin{document}

\maketitle

\section[template=wideslide,tocsection=false,slide=false]{Overview}

\begin{slide}[toc=,bm=]{Overview}
\tableofcontents[content=all,type=1]
\end{slide}

\section[template=wideslide]{Introduction}

\begin{slide}[toc=]{Cities, Paths and Tours}
\begin{itemize}
  \item
  Cities:
  \begin{equation}
    \mathcal{N} \coloneqq \mathcal{N}_n \coloneqq \left\{1,\ldots,n\right\}
    \quad (n \in \mathbb{N}, \; n \geq 2).
  \end{equation}
  \eqitspace%
  \item
  Paths:
  \begin{equation}
    \left(p_1,p_2,\ldots,p_m\right) \in \bigcup_{k \in \mathbb{N}}\mathcal{N}^k
    \eqqcolon \mathcal{P}_n;
  \end{equation}
  \eqitspace%
  \begin{itemize}
  \item
    non-empty;
  \item
    walking  length: $m-1$;
  \item
    end points: $p_1, p_m$;
  \item
    simple: $\lvert \left\{p_1,\ldots,p_{m-1}\right\} \rvert =
             \lvert \left\{p_2,\ldots,p_{m}\right\} \rvert = m-1$;
  \item
    closed: $p_1 = p_m, \; m > 1$;
    open otherwise;
  \item
    edge: open path of walking length 1;
  \item
    cycle: closed path;
  \item
    tour: simple cycle of walking length $n$,
    \begin{equation}\mathcal{T}_n \ldots \text{set of all tours (over }\mathcal{N}\text{)}.\end{equation}
  \end{itemize}
\end{itemize}
\end{slide}

\begin{slide}[toc=]{Path Operations}
\begin{itemize}
  \item
  Path concatenation:
  \begin{align}
    \begin{split}
    \oplus \; : & \; {\left( \mathcal{P}_n \cup 2^{\mathcal{P}_n} \right)}^2 \to \mathcal{P}_n \cup 2^{\mathcal{P}_n} \quad \text{(infix)},
    \\
    p \oplus q \coloneqq &
    \begin{cases}
      \left(p_1,\ldots,p_m,q_1,\ldots,q_k\right),
      & p = \left(p_1,\ldots,p_m\right) \in \mathcal{P}_n, \\
      & q = \left(q_1,\ldots,q_k\right) \in \mathcal{P}_n, \\
      & p_m \neq q_1;
      \\
      \left(p_1,\ldots,p_{m-1},q_1,\ldots,q_k\right),
      & p = \left(p_1,\ldots,p_m\right) \in \mathcal{P}_n, \\
      & q = \left(q_1,\ldots,q_k\right) \in \mathcal{P}_n, \\
      & p_m = q_1;
      \\
      \left\{p^\prime \oplus q \; \vert \; p^\prime \in p\right\},
      & p \notin \mathcal{P}_n, q \in \mathcal{P}_n;
      \\
      \left\{p \oplus q^\prime \; \vert \; q^\prime \in q\right\},
      & p \in \mathcal{P}_n, q \notin \mathcal{P}_n;
      \\
      \left\{p^\prime \oplus q^\prime \; \vert \; p^\prime \in p, q^\prime \in q\right\},
      & p \notin \mathcal{P}_n, q \notin \mathcal{P}_n
    \end{cases}
    \end{split}
  \end{align}
  (single reduction, notation extended to element-wise).
  \item Path reversal:
    \begin{equation}
      \operatorname{rev} : \mathcal{P}_n \to \mathcal{P}_n, \;
      \left(p_1,p_2,\ldots,p_{m-1},p_m\right) \mapsto \left(p_m,p_{m-1},\ldots,p_2,p_1\right).
    \end{equation}
\end{itemize}
%\board
\end{slide}

\begin{slide}{Traveling Salesman Problem}
\begin{itemize}
  \item
  Cost matrix: $C = \left(c_{i,j}\right) \in~\mathbb{R}^{n \times n},
    \quad n \in \mathbb{N} \left(n \geq 2\right)$.
  \item
  Path cost:
  \vspace{-2mm}
  \begin{equation}
    \omega_C: \mathcal{P}_n \to \mathbb{R}, \quad
    \left(p_1,p_2,\ldots,p_m\right) \mapsto \sum_{k=1}^{m-1} c_{p_k,p_{k+1}}.
  \end{equation}
  \eqitspace%
  \item
  Traveling Salesman Problem:
  given $M \subseteq \mathbb{R}^{n \times n}, T \subseteq \mathcal{T}_n, \; T \neq \varnothing$,
  \begin{equation}
    \operatorname{TSP}_{M,T} : M \to T, \; C \mapsto \argmin_{\tau \in T} \omega_C\left(\tau\right);
  \end{equation}
  $\argmin$ tour-valued, well-defined via \emph{some} total order on $\mathcal{T}_n$ (implicit argument).
  \item
  Variations:
  \begin{itemize}
  \item return cost along with (instead of) tour;
  \item $M, T$ as maps from $\mathbb{N}$;
  \item $(M, T)$: TSP case/class, omitted where justified;
  \item (s)TSP vs (a)TSP\@;
  \item here: $M \subseteq \mathbb{R}^{n \times n}_{\geq 0}$.
  \end{itemize}
  \vspace{-2mm}
  \item
  NP-hard in general case.
\end{itemize}
\end{slide}

\begin{slide}{Exponential Neighbourhoods \& Local Search}
\begin{itemize}
  \item
  Neighbourhood:
  \begin{equation}
    F: \mathcal{T}_n \to 2^{\mathcal{T}_n}
    %\setminus \varnothing;
    \quad \left(\text{we demand}\; \tau \in F\left(\tau\right) \; \forall \tau \in \mathcal{T}_n\right);
  \end{equation}
  exponential neighbourhood:
  $\left\vert F(\tau) \right\vert = \Omega(2^n) \qquad \forall \tau \in \mathcal{T}_n$.
  %\end{equation}
  \item
  Local Search:
  \begin{equation}
    l_C: \tau \mapsto \operatorname{TSP}_{M,F(\tau)}\left(C\right);
  \end{equation}
  sometimes: non-ascending ,,anytime'' heuristic.
  \item
  Iterative Local Search:
  \begin{align}
    \text{compute fixed point of}
    \quad
    \tau \mapsto
    \begin{cases}
      \tau \quad & \text{if} \; \omega_C\left(l_C\left(\tau\right)\right) = \omega_C\left(\tau\right),
      \\
      l_C\left(\tau\right) & \text{otherwise};
    \end{cases}
  \end{align}
  termination conditions may vary.
\end{itemize}
\board
\end{slide}

\begin{slide}{Tours and Permutations}
\begin{itemize}
  \item
  Associated permutation:
  \begin{align}
  \begin{split}
    \sigma_{\Cdot} :
    \mathcal{T}_n \to \; & \mathcal{S}_n, \quad
    \tau = \left(p_1, p_2, \ldots, p_n, p_1 \right)
    \mapsto
  \sigma_\tau \coloneqq
  \begin{pmatrix}
    1 & 2 & \cdots & n \\
    p_1 & p_2 & \cdots &  p_n
  \end{pmatrix},
  \\
  & \mathcal{S}_n \ldots \text{symmetric group (on }\mathcal{N}_n \text{)}.
  \end{split}
  \end{align}
\eqitspace%
\item
  Equivalence:
  \begin{align}
    \tau \sim \tau^\prime
    %\Leftrightarrow
    \;:\;
    &
    \sigma_\tau = \sigma_{\tau^\prime} \circ \rho
    \quad \text{for some}\; \rho \in \left<\left(1 \; 2 \; \cdots \; n \right)\right>;
    \\
    \tau \sim \tau^\prime
    \;\rightarrow\;
    &
    \omega_C\left(\tau\right) = \omega_C\left(\tau^\prime\right)
    \quad \forall C \in \mathbb{R}^{n \times n}.
  \end{align}
  \eqitspace%
  \item
  Reduction $\mathcal{T}_n \to \mathcal{T}_n/{\sim}$ (e.g.\ single starting city $p_1$) possible.
\end{itemize}
\end{slide}

\begin{slide}[toc=,bm=]{Recap}
\tableofcontents[content=currentsection,type=1]
\end{slide}

\section[template=wideslide]{Theory}

\begin{slide}[toc=Pyramidal Tours]{Motivation: Pyramidal Tours}
\begin{itemize}
  \item
  Pyramidal path:
  \begin{align}
  \text{simple} \; \left( p_1, \ldots, p_k, q_1, \ldots, q_m \right) \; \text{with} \;
  \begin{cases}
  k+m \geq 1, & \\
  p_i < p_{i+1} \; & \forall i \in \mathcal{N}_{k-1},\\
  q_j > q_{j+1} \; & \forall j \in \mathcal{N}_{m-1}.
  \end{cases}
  \end{align}
  %\eqitspace%
  \item
  All pyramidal tours: neighbourhood\footnote{justified later} $\operatorname{Pyr}_n$ of
  $\tau^\ast = \left(1, 2, \ldots, n, 1\right)$.
  \begin{figure}[H]
    \centering
    \includegraphics[width=0.42\textwidth]{../plot/build/pyr-2.eps}
    \caption{%
      For a pyramidal tour $\tau$,
      connected plot of $\sigma_\tau$'s
      graph resembles a pyramid (here: $n = 5$).
    }
  \end{figure}
\end{itemize}
\end{slide}

\begin{slide}[toc=]{Pyramidal Tours: Recurse!}
  \begin{figure}[H]
    \centering
    \includegraphics[width=0.42\textwidth]{../plot/build/pyr-3.eps}
    \caption{$\operatorname{Pyr}_n$: city-at-a-time construction.}
  \end{figure}
\begin{itemize}
  \item
    Construction:
  \begin{align}
    \begin{split}
    V\left(i,j\right) & : \mathcal{N}_n^2 \to 2^{\mathcal{P}_n}, \\
    \left(i,j\right) & \mapsto \; \text{all pyramidal paths} \left(i,p_1,\ldots,p_m,j\right)
    \\
    & \qquad \text{with} \; \left\{ p_1,\ldots,p_m \right\} = \left\{ k,\ldots,n \right\},
    \; \text{where} \; k = \max\left\{i,j\right\}+1.
    \\
    \Rightarrow \;
    V\left(i,j\right) & =
    \begin{cases}
      \left\{\left(i,j\right)\right\}, & n \in \left\{i,j\right\},
      \\
      \left(i\right) \oplus V\left(k,j\right)
      \;\bigcup\;
      V\left(i,k\right) \oplus \left(j\right)
      & \text{otherwise ($k$ as above)};
    \end{cases}
    \\
    V\left(1,1\right) & \ldots \text{all pyramidal tours in} \; \mathcal{T}_n.
    \end{split}
  \end{align}
\end{itemize}
\end{slide}

\begin{slide}[toc=]{Pyramidal Tours: Dynamic Programming Solution}
\begin{itemize}
  \item
    Best pyramidal tour:
    \begin{align}
    \begin{split}
      \Phi_C\left(i,j\right) & \coloneqq \; \text{lowest cost path in} \; V\left(i,j\right).
    \\
    \Rightarrow \;
    \Phi_C\left(i,j\right) & =
    \begin{cases}
      \left(i,j\right), & n \in \left\{i,j\right\}, \\
      \argmin \omega_C \text{ over }
      \left\{
      \left(i\right) \oplus \Phi_C\left(k,j\right)
      ,
      \Phi_C\left(i,k\right) \oplus \left(j\right)
      \right\}
      & \text{o.w.\ ($k$ a.a.)};
    \end{cases}
    \\
    \Phi_C\left(1,1\right) & = \operatorname{TSP_{\mathbb{R}^{n \times n},\operatorname{Pyr}_n}}\left(C\right).
    \end{split}
  \end{align}
\end{itemize}
\vspace{-3mm}
\begin{figure}[H]
  \centering
  \includegraphics[width=0.58\textwidth]{../plot/build/trace_apyr.eps}
  \caption{Pyramidal aTSP: recursion tree ($n = 5$).}
\end{figure}
\end{slide}

% FIXME: how to escape comma?
\begin{slide}[toc=Dynamic Programming${,}$ Recursion and Complexity]{Pyramidal Tours: Quadratic Time}
  \begin{itemize}
  \item
  Dynamic programming solution $\rightarrow$ DFS.
  \item
  Quadratic time assuming:
  \begin{itemize}
    \item memoization (sic!): compute subset of $\Phi$'s graph;
    \item constant time memory access (practice: good hash not always obvious);
    \item constant time tour (re)construction;
    \item constant time tour comparison (compute cost in $\Phi$).
  \end{itemize}
  \item
  BFS: linear space overhead (tail recursion).
  \item
  Bottom-up computation:
  \begin{itemize}
    \item
    none better asymptotically;
    \item
    referential transparency becomes bulky $\rightarrow$ error-prone;
    \item
    not always feasible (see next section).
  \end{itemize}
  \end{itemize}
\board
\end{slide}

\begin{slide}[toc=Strongly Balanced Tours]{Strongly Balanced Tours: Motivation}
  \begin{itemize}
  \item
  Four-point conditions: restrict set of cost matrices $\left(c_{i,j}\right)$.
  \item
    Demidenko condition:
    \begin{equation}
      c_{i,j} + c_{k,l} \leq c_{i,k} + c_{j,l} \quad
      \forall \; 1 \leq i < j < k < l \leq n
    \end{equation}
    $\Rightarrow$
    TSP pyramidally solvable.
  \item
    Relaxed Supnick TSP:
    \begin{equation}
      c_{i,k} + c_{j,l} \leq c_{i,l} + c_{j,k} \quad
      \forall \; 1 \leq i < j < k < l \leq n
    \end{equation}
    NP-hard, symmetric.
    \item
    Strongly balanced tours:
    \begin{itemize}
      \item subset of RS-TSP;
      \item polynomially solvable;
      \item first implementation: herein.
    \end{itemize}
  \end{itemize}
\end{slide}

\begin{slide}[toc=]{Strongly Balanced Tours: Construction}
  \begin{itemize}
    \item
    Partial path: set of (unoriented) paths $\left[a,b\right] = \left(a,\ldots,b\right)$.
    \item
    Recursive construction:
    \begin{align}
      \begin{split}
      \mathcal{B}_1 = & \left\{ \left\{\left(1\right)\right\} \right\};
      \\
      \mathcal{B}_m = & \bigcup_{\tau \in \mathcal{B}_{m-1}}
      %\mathcal{B}_1 = \left\{ \left\{\left(1\right)\right\} \right\};
      %\qquad
      %\mathcal{B}_m = \bigcup_{\tau \in \mathcal{B}_{m-1}}
        \left(
        %\left\{
        \operatorname{add}\left(\tau,m\right)
        \cup
        \operatorname{append}\left(\tau,m\right)
        %\right\}
        \cup
        \operatorname{merge}\left(\tau,m\right)
        \right),
      \end{split}
    \end{align}
    where for
        $ \tau = \left\{ \pi_1, \pi_2, \ldots \right\} $
        with
        $
        \pi_i = \left[m_i,m_i^\prime\right], \;
        m_i \leq m_i^\prime%\footnote{reverse where needed}
        \; \forall i, \;
        m_1 < m_2 < m_3 < \ldots ,
        $
    \begin{align}
      \begin{split}
        \operatorname{add}\left(\tau,m\right) & \coloneqq \tau \cup \left\{ \left(m\right) \right\},
        \\
        \operatorname{append}\left(\tau,m\right) & \coloneqq \tau \setminus \pi_1 \cup \left\{ \left(m\right) \oplus \pi_1 \right\},
        \\
        \operatorname{merge}\left(\tau,m\right) & \coloneqq
        \begin{cases}
          \varnothing, & \lvert\tau\rvert < 2,
          \\
          \tau \setminus \left\{\pi_1,\pi_2\right\} \cup \left\{ \operatorname{rev}\left(\pi_1\right) \oplus \left(m\right) \oplus \pi_2 \right\} & o.w..
        \end{cases}
      \end{split}
    \end{align}
    %\eqitspace%
    \vspace{-3mm}
  \item
    Strongly balanced tours:
    \begin{equation}
      %\tau \oplus \left(\sigma_\tau\left(1\right)\right) \in \mathcal{T}_n : \left\{\tau\right\} \in \mathcal{B}_n,
      \overline{\tau} \in \mathcal{T}_n : \left\{\tau\right\} \in \mathcal{B}_n,
      \qquad
      \text{where} \;
      \overline{\tau} \coloneqq \tau \oplus \left(\sigma_\tau\left(1\right)\right);
    \end{equation}
    neighbourhood\footnote{justified later} of $\tau^\ast =
      \left( \ldots \; 5 \; 3 \; 1 \; 2 \; 4 \; \ldots \right).$
  \end{itemize}
  %\board
\end{slide}

\begin{slide}[toc=]{Strongly Balanced Tours: Dynamic Programming Solution}
    \begin{figure}[H]
      \centering
      \includegraphics[width=0.63\textwidth]{../plot/build/bal-1.eps}
      \caption{
        Strongly balanced path construction options: choose $m$ to be adjacent to $m_1$, both $m_1$ and $m_2$, or none of the two.
       %For tours, we make the additional requirement for $n$ to be added to single paths only.
     }
    \end{figure}
  \begin{itemize}
  \item
    Recursive construction $\rightarrow$ dynamic programming solution (analogous to pyramidal case).
    \begin{itemize}
    \item
      S.\ B.\ recursion tree node:
      \begin{equation}
      \left(m, \;
      \left\{\left\{a_i,b_i\right\} \mid i \in I\right\}\right)
       \in \mathcal{N}_n \times 2^{\mathcal{N}_n^2}.
      \end{equation}
      \eqitspace%
    \item
      Solution yields, for each node,
      \vspace{2mm}
      \begin{quote}
      {\itshape
        a minimum cost set of paths
        through all cities larger than $m$ and all cities $a_i$,$b_i$,
        with end points given by the non-singleton sets
        $\left\{a_i,b_i\right\}$,
        %belonging to a single element of $\mathcal{B}_n$
        %all reachable through a single $\operatorname{add}$, $\operatorname{append},$
        %or $\operatorname{merge}$ of $m+1$,
        which, per path concatenation,
        complements the corresponding element of $\mathcal{B}_m$
        to a tour corresponding to a singleton element of $\mathcal{B}_n$.
        %complements said sets (seen as paths) to a single cycle
        %and  subpaths of (can be extended to) a single strongly balanced tour.
      }
      \end{quote}
      \vspace{2mm}
    \end{itemize}
    \eqitspace%
  \item
    Number of intervals bounded by constant $M$ $\;\Rightarrow\;
    \mathcal{O}\left(n^{2M+1}\right)$ time complexity.
  \end{itemize}
\end{slide}

\begin{slide}[toc=]{Strongly Balanced Tours: Recursion Tree}
  \begin{figure}[H]
    \centering
    \includegraphics[width=1.0\textwidth]{../plot/build/trace_bal.eps}
    \caption{Strongly balanced (s)TSP: recursion tree ($M=2,n=8$).}
  \end{figure}
\end{slide}

\begin{slide}[toc=]{Strongly Balanced Tours: Linear Time Conjecture}
  \begin{itemize}
  \item
    Node type:
    \begin{equation}
    \chi : \left(m,\left\{\left\{a_i,b_i\right\}\right\}\right)
    \mapsto
     \left\{\left\{m-a_i,m-b_i\right\}\right\}.
   \end{equation}
   %\eqitspace%
   when evaluated along recursion tree: does not depend on $m$!
   \item
   Conjecture: number of node types in recursion tree limited
   $\rightarrow \; \mathcal{O}\left(n\right)$ time.
    %{\small
    \begin{table}[htpb]
    \centering
    \begin{tabular}{cSc}
      \toprule
      {\bfseries $M$ } &
      {\bfseries Node types} &
      {\bfseries Recursion depth } \\
      \midrule
      2 &     16 &  7 \\
      3 &    121 & 13 \\
      4 &   1074 & 20 \\
      5 &  10387 & 28 \\
      6 & 107176 & 37 \\
      \bottomrule
    \end{tabular}
    \caption{Total number of nodes encountered by depth first search.}
    \end{table}
    %}
  \end{itemize}
  %\board
\end{slide}

\begin{slide}[toc=Tour Sets as Neighbourhoods]{Tours and Permutations: Tour Sets as Neighbourhoods}
  \begin{itemize}
  \item
  Associated permutation:
  \begin{align}
  \begin{split}
    \sigma_{\Cdot} & :
    \mathcal{T}_n \to \; \mathcal{S}_n, \quad
    \tau = \left(p_1, p_2, \ldots, p_n, p_1 \right)
    \mapsto
  \sigma_\tau \coloneqq
  \begin{pmatrix}
    1 & 2 & \cdots & n \\
    p_1 & p_2 & \cdots &  p_n
  \end{pmatrix},
  \\
  \pi & \coloneqq \left(\sigma_{\Cdot}\right)^{-1}.
  \end{split}
  \end{align}
  \eqitspace%
  \item
    Tour sets as neighbourhoods:
    \begin{equation}
      \tau^\ast \in T \subseteq \mathcal{T}_n
      \; \rightarrow \;
      \text{neighbourhood} \;
      F_T : \tau \mapsto
      \pi \left(
        \left\{ \sigma_{\tau} \circ \sigma_{\tau^\ast}^{-1} \circ \sigma_{\tau^\prime} \mid \tau^{\prime} \in T\right\}
      \right);
    \end{equation}
    i.e.\ $T$ as permutations of $\tau^\ast \; \rightarrow $ permutations of $\tau$.
  \end{itemize}
  \vspace{-2mm}
  \begin{figure}[H]
    \centering
    \includegraphics[width=0.45\textwidth]{../plot/build/ext-1.eps}
    \caption{%
      Tour sets as neighbourhoods:
      $\tau \in F_T\left(\tau\right) \; \forall \tau \in \mathcal{T}_n, \; F_T\left(\tau^\ast\right) = T$ hold true.
    }
  \end{figure}
\end{slide}

\begin{slide}{Local Search and Flowers}
  \begin{itemize}
  \item
    Flower: extend set of tours $T$ via permutations in $S \subseteq \mathcal{S}_n$:
    \begin{equation}
      \mathcal{F}\left(S,T\right) \coloneqq \pi\left(\left\{ s \circ \sigma_\tau \mid s \in S, \tau \in T\right\}\right) \cup T;
    \end{equation}
  choose best tour from $\lvert S \rvert+1$ local search results.
  \item
    Pyramidal ILS:
    \begin{equation}
      F_{\mathcal{F}\left(S,T\right)}, \;
      T = \operatorname{Pyr}_n, \;
      S = \left<\sigma^\ast\right>, \;
      \sigma^\ast = \left(1\;2\;\cdots\;n\right).
    \end{equation}
  \eqitspace%
  \item
  Strongly balanced ILS:
    \begin{equation}
      \text{e.g.} \; S = \left<\sigma^\ast\right>\mu \;,\;
      \mu =
      \begin{pmatrix}
        1 & \cdots & n \\
        \ldots & \; 5 \; 3 \; 1 \; 2 \; 4 \; & \ldots
      \end{pmatrix};
    \end{equation}
    better set possible?
  \end{itemize}
  \board
\end{slide}

\begin{slide}[toc=,bm=]{Recap}
\tableofcontents[content=currentsection,type=1]
\end{slide}

\section[template=wideslide]{Implementation}

\begin{slide}[toc=Key Features]{Implementation: Key Features}
  \begin{itemize}
  \item
  Standard ML w/ smlnj-lib.
  \item
  Usable via
  \begin{itemize}
    \item REPL;
    \item standalone executable;
    \item shared library.
  \end{itemize}
  \item
  Builds via MLton, Poly/ML, SML/NJ.
  \item
  Tested under Linux/{\ttfamily amd64}, Windows/{\ttfamily x86}.
  \item
    Times measured here:
    Linux 3.10.17/{\ttfamily amd64} @ i5-2520M.
  \item
  Available under
  %\begin{quote}
    %\hspace{3.4em}
    \url{http://bitbucket.org/mad_hatter/rstsp/}.
  %\end{quote}
  \end{itemize}
\end{slide}

% TODO: use onslide?
\begin{slide}[toc=Benchmarks: Time]{Measured Time: Local Search}
  \begin{figure}[H]
    \centering
      \includegraphics[width=0.87\textwidth]{../plot/build/mlton_time_random_steady.eps}
  \end{figure}
\end{slide}
\begin{slide}[toc=]{Measured Time: Local Search}
  \begin{figure}[H]
    \centering
      \includegraphics[width=0.87\textwidth]{../plot/build/mlton_time_random_fast.eps}
  \end{figure}
\end{slide}
\begin{slide}[toc=]{Measured Time: Local Search}
  \begin{figure}[H]
    \centering
      \includegraphics[width=0.87\textwidth]{../plot/build/mlton_time_random_low.eps}
  \end{figure}
\end{slide}
\begin{slide}[toc=]{Local Search: Pyramidal vs S.\ B.\ Time}
  \begin{figure}[H]
    \centering
      \includegraphics[width=0.87\textwidth]{../plot/build/mlton_time_random_med.eps}
  \end{figure}
\end{slide}
\begin{slide}[toc=]{Local Search: Pyramidal vs S.\ B.\ Time}
  \begin{figure}[H]
    \centering
      \includegraphics[width=0.87\textwidth]{../plot/build/mlton_time_random_hi.eps}
  \end{figure}
\end{slide}

\begin{slide}[toc=Benchmarks: Solutions]{Tour Quality: Synthetic Results}
  \begin{figure}[H]
    \centering
    \includegraphics[width=0.86\textwidth]{../plot/build/random_val.eps}
    \caption{Tour Quality: local search and heuristics.}
  \end{figure}
\end{slide}

\begin{slide}[toc=]{Tour Quality: TSPLIB Benchmarks}
  \begin{figure}[H]
    \centering
    \includegraphics[width=0.74\textwidth]{../plot/build/tsplib_val.eps}
    \caption{Benchmark: 25 TSPLIB instances, sizes $\leq 130$.}
  \end{figure}
\end{slide}

\begin{slide}[toc=]{Measured Time: TSPLIB Benchmarks}
  \begin{figure}[H]
    \centering
    \includegraphics[width=0.74\textwidth]{../plot/build/tsplib_time.eps}
    \caption{TSPLIB instances: time needed.}
  \end{figure}
\end{slide}

\section[template=wideslide]{Conclusions}

\begin{slide}[toc=Recap]{Recap}
\tableofcontents[content=all,type=0]
\end{slide}

\begin{slide}{Areas for Further Research}
{\small

  \begin{itemize}
  \item
  Open Tasks:
    \begin{itemize}
      \item implement BFS
      \item precompute flowers (guarantees uniqueness)
      \item generate \& benchmark RS-TSP instances;
      \item confirm complexity conjecture for larger sizes.
    \end{itemize}
  \item
  Open Questions:
    \begin{itemize}
    \item better LS extension for S.\ B. tours;
    \item combine with other neighbourhoods / starting tours?
    \item other metrics?
    \item pyramidal search: test on aTSP;
    \item \ldots
    \end{itemize}
  \item
  Concurrency and Computation Models:
    \begin{itemize}
    \item tuple spaces (Linda, Clotilde, \ldots)
    \item lightweight (a)syncronous models (Mythryl, Clojure, Erlang, AliceML \ldots)
    \item \ldots
    \end{itemize}
  \end{itemize}

}
\end{slide}

%\begin{note}{Note}
%  Some note.
%\end{note}

\section[template=wideslide]{Your Questions}

\end{document}
\endinput
