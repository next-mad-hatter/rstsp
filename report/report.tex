%
% $Id$
% $Author$
% $Date$
% $Rev: 34 $
%

%\documentclass[titlepage,index=totoc,bibliography=totoc]{scrartcl}
\documentclass[index=totoc,bibliography=totoc]{scrartcl}
\KOMAoptions{fontsize=11pt}
%\KOMAoptions{BCOR=0.25mm}
%\usepackage[left=3.1cm, right=3.1cm]{geometry}

\usepackage{mathdesign}
\usepackage[no-math]{fontspec}
\defaultfontfeatures{Ligatures=TeX}
\setmainfont[Numbers=OldStyle]{Latin Modern Roman}
%\setmainfont[Numbers=OldStyle]{CMU Serif}
%\setmainfont[Numbers=OldStyle]{Gentium}
%\setmainfont[Numbers=OldStyle]{EB Garamond}
\setsansfont[Numbers=OldStyle]{CMU Sans Serif}
%\setsansfont{Droid Sans}
%\setmonofont{CMU Typewriter Text}
\setmonofont[Scale=0.9]{Droid Sans Mono}
%\setmonofont{DejaVu Sans Mono}
\newfontfamily\CodeFont[Scale=0.85]{DejaVu Sans Mono}
%\newfontfamily\CodeFont[Scale=0.85]{Consolas}
%\newfontfamily\CodeFont[Scale=0.85]{Anonymous Pro}

\usepackage{luatextra}

\usepackage[english,ngerman,russian]{babel}
\usepackage{csquotes}

% as of now, xindy is broken on my machine right
\usepackage{imakeidx}
\indexsetup{level=\chapter,toclevel=chapter}
\makeindex

\usepackage[backend=biber,style=alphabetic]{biblatex}
%\usepackage[fixlanguage]{babelbib}
\addbibresource{references.bib}
\usepackage[nottoc]{tocbibind}

\usepackage{amsmath}
%\usepackage{unicode-math}
\usepackage{amsfonts}
\usepackage{amssymb}

\usepackage{commath}
\usepackage{mathtools}

\usepackage{amsthm}
\usepackage{thmtools}
\usepackage{etoolbox}
\usepackage{changepage}

\declaretheorem[style=definition,name=Definition]{define}
\declaretheorem[style=plain]{theorem}
\declaretheorem[sibling=theorem]{lemma}
\declaretheorem[style=remark]{remark}
%\BeforeBeginEnvironment{theorem}{\begin{adjustwidth}{1cm}{1cm}}
%\AfterEndEnvironment{theorem}{\endadjustwidth}
%\AfterEndEnvironment{define}{\noindent\ignorespaces}
%\makeatletter
%\def\@endtheorem{\endtrivlist}%
%\makeatother

\usepackage{caption}
\usepackage{ragged2e}

\usepackage{graphicx}
%\usepackage[export]{adjustbox}
%\usepackage{epstopdf}
%\epstopdfDeclareGraphicsRule{.tif}{png}{.png}{convert #1 \OutputFile}
%\AppendGraphicsExtensions{.tif}
\usepackage{luamplib}
\usepackage{etoolbox}
\preto\mplibcode{\center\leavevmode}
\appto\endmplibcode{\endcenter}
\usepackage{wrapfig}
\usepackage{float}
\usepackage{subfig}
\usepackage{sidecap}
\usepackage[bottom]{footmisc}

\usepackage{xcolor}
\definecolor{Gray}{gray}{0.9}
\definecolor{blue}{rgb}{0,0,0.3}
\definecolor{green}{rgb}{0,0.3,0}
\definecolor{codeblue}{rgb}{0,0,0.3}
\definecolor{codegreen}{rgb}{0,0.3,0}
\definecolor{codered}{rgb}{0.3,0,0}

\usepackage{listings}
\usepackage{needspace}
\lstset{%
  basicstyle=\CodeFont\footnotesize, %\footnotesize\ttfamily,
  frame=single, %line/leftline/single
  aboveskip=\topsep,
  belowskip=\topsep,
  breakatwhitespace=true,
  showstringspaces=false,
  breaklines=true,
  commentstyle=\itshape\color{codegreen},
  keywordstyle=\bfseries\color{codeblue},
  stringstyle=\color{codered},
  captionpos=t\/
}
\captionsetup[lstlisting]{font=footnotesize}
\newcommand{\mylisting}[2][]{%
  \lstinputlisting[caption={\texttt{\detokenize{#2}}},#1]{#2}%
}

%\usepackage{kantlipsum}
%\usepackage{blindtext}
\usepackage{marginnote}
\newcommand{\annotation}[1]{\marginpar{\small\itshape\color{green}#1}}

\usepackage{booktabs}
\usepackage{longtable}
\usepackage{color, colortbl}
\usepackage{multirow}
\usepackage{varwidth}
\usepackage{pgfplotstable}

\usepackage[title,page,titletoc]{appendix}
\newcommand*{\Appendixautorefname}{Appendix}
\usepackage{cleveref}

\usepackage{verbatim}
\usepackage{cprotect}

\usepackage{afterpage}
%\makeatletter
%\def\verbatim@font{\normalfont\itshape}
%\makeatother
\newcommand\blankpage{%
  \null%
  \thispagestyle{empty}%
  %\addtocounter{page}{-1}%
  \vspace*{\fill}
  \centering
  {\em {\small
  This page would have been intentionally left blank had we not chosen to
  \href{https://en.wikipedia.org/wiki/Recursion}{mention it}.
  }}%
  %\verbatiminput{|"/usr/games/fortune"}
  \vspace{\fill}%
  \newpage
}

\usepackage{abstract}

%\usepackage[toc]{multitoc}
\usepackage[subfigure]{tocloft}
\renewcommand{\cftsecleader}{\cftdotfill{\cftdotsep}}

\DeclareMathOperator*{\argmin}{arg\,min}
\DeclareMathOperator*{\argmax}{arg\,max}

\usepackage[pdfauthor={M. Deineko},
            pdfproducer={lualatex},
            pdfcreator={LaTeX},
            colorlinks=true,
            linkcolor=blue,
            urlcolor=blue,
            citecolor=blue,
            linktoc=section,
            unicode=true
            ]{hyperref}
\hypersetup{pdftitle=Two Exponential Neighbourhoods}

\let\defstyle\itshape
%\def\defstyle{\itshape}

\begin{document}

\selectlanguage{english}

\title{Two Exponential Neighbourhoods}
\subtitle{Pyramidal and strongly balanced traveling salesman problem and heuristics --- theory and implementation}
% FIXME: supervisor format ?
% FIXME: title pic / multitoc ?
% FIXME: twosided / book format ?
%
% $Id$
% $Author$
% $Date$
% $Rev: 37 $
%

\newcommand{\mailto}[1]{\href{mailto:#1}{\nolinkurl{#1}}}

\author{%
 Maksym~Deineko \\ {\small \mailto{max.deineko@gmail.com} }
 \and
 Supervisor: Eranda~Dragoti-Cela, Ao.~Univ.-Prof.\ Dipl.-Ing.\ Dr.techn. \\ {\small \mailto{cela@math.tugraz.at} }
}

\date{\today}

% FIXME: titlepage / spacing ?
\maketitle

% FIXME: better wording / tours family ?
\begin{abstract}
  Pyramidal tours constitute a well known polynomially solvable case of TSP\@.
  In this paper we present a first implementation of a related tours family,
  show extension of both solutions to local search heuristics and compare results thus obtained.
\end{abstract}

\thispagestyle{empty}

%\setlength{\cftbeforetoctitleskip}{-2em}
\tocloftpagestyle{empty}
\afterpage{\blankpage}
\clearpage
\tableofcontents

%\setlength{\parindent}{0cm}

\clearpage
\dictum[Sherlock Holmes]{Nothing clears up a case so much as stating it to another person.}
\section{Introduction}

The {\defstyle traveling salesman problem (TSP)} can be colloquially stated as follows:
\index{TSP|see{Traveling Salesman Problem}}
\blockquote{%
  \index{Traveling Salesman Problem|(}
  A traveling merchant (or salesman) wishes to visit each city from a given
  list exactly once, starting from and returning to a select one of the cities
  given.  Assuming that distances between the cities are known and not
  subject to change, what would be the shortest route for him to take?
  \index{Traveling Salesman Problem|)}
}
This well studied problem in combinatorial optimization is known to be
NP-hard \cite{Karp72}; hence, one is interested in efficiently solvable
cases as well as useful heuristics, of which a good number exists.

In this paper we look at two large sets of feasible solutions over which
the TSP can be solved in polynomial time and extension of these to
iterative heuristics.  For this, we need to establish some basic
definitions first.

\subsection{Definitions and notation}

The TSP, as most mathematical problems, can be modeled in a number of often
equivalent or similar ways, depending on what aspects of the problem are
relevant to the task at hand --- which, for a problem as extensively studied
as this one, results in a great number of common terms which from author to
author can differ ever so slightly.

Hence we need to establish a set of definitions which we shall try to keep
precise enough to be able to convey this paper's content clearly,
while also simple enough so as not to distract from it.

\begin{define}[Cities, paths and tours]
Given some $n \in \mathbb{N} \left(n \geq 2\right)$,
let \[\mathcal{N} \coloneqq \mathcal{N}_n \coloneqq \left\{1,\ldots,n\right\}.\]
We will refer to elements of $\mathcal{N}$ as {\defstyle cities}.
A non-empty tuple $\left(p_1,p_2,\ldots,p_m\right)$
of cities is called a {\defstyle path} (of {\defstyle length} $m-1$ {\defstyle over} $\mathcal{N}$);
it is called {\defstyle simple} if it contains distinct entries only, with
possible exception of its {\defstyle end points} $p_1$ and $p_m$, i.e. if
\[\left|\left\{p_1,p_2,\ldots,p_{m-1}\right\}\right| =
\left|\left\{p_2,p_3,\ldots,p_{m}\right\}\right| = m-1.\]  We shall refer to
a path with distinct end points as an {\defstyle open} path, one with
equal end points --- a {\defstyle closed} one, or a {\defstyle cycle}.
An open path of length $1$ is also called an {\defstyle edge};
a simple cycle of length $n$ is called a {\defstyle tour}.
We additionally denote by $\mathcal{T}_n$ set of all tours over $\mathcal{N}_n$.
\end{define}

\begin{define}[Distances and cost]
Given a $n \in \mathbb{N} \left(n \geq 2\right)$ and a matrix $C =
\left(c_{ij}\right) \in ~ \mathbb{R}^{n \times n}$, which we shall call
{\defstyle cost} (or {\defstyle distance) matrix}, we will refer to its
elements as {\defstyle distances} or {\defstyle weights}.
Now let \[
%  \operatorname{len} A \xrightarrow[\pi \mapsto \operatorname{len}(\pi)]{} B.
%  \operatorname{len}_C: \bigcup_{k \in \mathbb{N}, k \geq 2}\mathcal{N}^k \to \mathbb{R}, \quad
  \mathfrak{w}_C: \bigcup_{k \in \mathbb{N}, k \geq 2}\mathcal{N}^k \to \mathbb{R}, \quad
  \left(p_1,p_2,\ldots,p_m\right) \mapsto \sum_{k=1}^{m-1} c_{p_k p_{k+1}}.
\]
For a path $\pi$ over $\mathcal{N} = \mathcal{N}_n$ we then call $\mathfrak{w}_C\left(\pi\right)$ {\defstyle cost} of $\pi$.
An edge's cost is also called its {\defstyle weight} (which agrees with the definition above).
\end{define}

\begin{define}[Traveling salesman problem]
  Given a $n \in \mathbb{N} \left(n \geq 2\right)$,
  $M \subseteq~\mathbb{R}^{n \times n}$ and
  $T \subseteq \mathcal{T}_n$, we define {\defstyle traveling salesman problem}
  ({\defstyle over $(M,T)$}, also {\defstyle TSP}) to be the map
\[
  \operatorname{TSP}_{M,T}: M \to T, \quad
  C \mapsto \argmin_{\tau \in T} \mathfrak{w}_C\left(\tau\right).
\]
We call $n$ the TSP's {\defstyle dimension}.
If $M$ is a subset of symmetric matrices (over $\mathbb{R}$), the TSP is called {\defstyle symmetric (sTSP)},
otherwise --- {\defstyle asymmetric (aTSP)}.
\end{define}

\begin{remark}
Frequently, the TSP is defined via a map which provides the sets $M$ and $T$ for every dimension $n$.
We make no such requirement --- however, such an extension is usually easily established from context.
\end{remark}

\begin{remark}
Also not unusual is a variation of TSP which returns the cost of optimal tour along with or instead of such a tour;
transitions between these definitions are usually trivial.
\end{remark}

\begin{remark}
Whenever the sets $M,T$ from above definition are established elsewhere, they are usually omitted from notation.
\end{remark}

\begin{remark}
Often, the TSP is defined to be the symmetric version
\annotation{Reduction remark necessary / wording?}
(aTSP can also be reduced to sTSP of double dimension).
We shall make no such assumption without a prior notice.
\end{remark}

\begin{remark}
Since $\operatorname{TSP}_{T,M}\left(C\right) = \operatorname{TSP}_{T,M}\left(C + \lambda J\right)$
holds true for all $\lambda \in \mathbb{R}$ and $C \in M$,
where $J$ is matrix of ones (of suitable dimension), we can always assume
\[
  c_{ij} \geq 0 \quad \forall i,j \in \mathcal{N}
\]
and shall do so henceforth.
\end{remark}

We will make extensive use of path {\defstyle concatenation}, which we define to be the following map (used as infix operator):
\begin{define}
\[
  \oplus: {\left( \bigcup_{k \in \mathbb{N}}\mathbb{N}^k \right)}^2 \to \bigcup_{k \in \mathbb{N}}\mathbb{N}^k,
  \quad \left(\left(r_1,\ldots,r_l\right), \left(s_1,\ldots,s_m\right)\right) \mapsto \left(r_1,\ldots,r_l,s_1,\ldots,s_m\right).
\]
\end{define}

We will also sometimes need to view tours as maps on $\mathcal{N}$ or
$\mathbb{R}^{n \times n}$.  For this, we note that there is a natural
bijection between tours in $\mathcal{T}_n$ and permutations on
$\mathcal{N}_n$ (or, by same argument, paths of length $n-1$).
\begin{define}
Given a set $\mathcal{N}$ of $n$ cities, let $\mathcal{S}_n$ be the symmetric group (on $\mathcal{N}$).
We define $\mathfrak{s}_{\bullet}$ (using argument-in-subscript notation for reasons of legibility) to be the map
\[
  \mathcal{T}_n \to \mathcal{S}_n,
  \quad \tau = \left(p_1,p_2,\ldots,p_n,p_1\right) \mapsto
  \mathfrak{s}_\tau \coloneqq
  \left( \begin{smallmatrix}
    1 & 2 & \cdots & n \\
    p_1 & p_2 & \cdots &  p_n
  \end{smallmatrix} \right),
\]
which is clearly bijective.  We shall call its inverse $\mathfrak{t}_{\bullet}$ (again, in argument-in-subscript notation),
reserving the notation $\mathfrak{s}_{\tau}^{-1}$ for the
permutation inverse to $\mathfrak{s}_{\tau}$.
A tour $\tau = \mathfrak{t}_{\mathfrak{s}_{\tau}}$ and permutation $\mathfrak{s}_{\tau}$ are said to be {\defstyle associated} with each other.
\end{define}

\begin{remark}
  \annotation{TODO}
  Cost invariance under circular permutations, action of $\mathcal{S}_n$ on $\mathbb{R}^{n \times n}$.
\end{remark}

\subsection{Exponential neighbourhoods and local search heuristics}

\clearpage
\dictum[L. Peter Deutsch]{To iterate is human, to recurse divine.}
%\dictum[Honore de Balzac]{To get a traveling salesman drunk is the height of impossibility.}
\section{Two Neighbourhoods}

\subsection{Pyramidal tours}

\annotation{Who was first?}
Pyramidal tours constitute a classic example of a TSP class where an
optimum solution for an exponential tours set (also, a neighbourhood) can
be found in polynomial time \cite{bellman,gilmore}.

\begin{define}
  A path $\pi = \left( k_1, \ldots, k_m \right)$ is called {\bfseries simple}
  if it consists only of distinct nodes, except, possibly, end nodes $k_1$ and $k_m$.
  In case where $k_1 = k_m$ we shall call such a path {\bfseries closed}.
\end{define}
\begin{define}[Pyramidal Tour]
  A simple path $\pi = \left( k_1, \ldots, k_p, l_1, \ldots, l_q \right), \; p+q \geq 1,$
  is called {\bfseries pyramidal} if
  $k_i < k_j, l_i > l_j \; \forall i < j$.
  A (simple) closed path is called a {\bfseries tour}.
\end{define}
\noindent
Since all tours associated with $S_n$ have the form $\left(1, k_1, \ldots,
k_p, n, l_1, \ldots, l_q, 1 \right)$, it follows that the number of all
such tours is $2^{n-2}$.  Also note how the graph of such a permutation
considered a function ${\mathbb Z}_n \rightarrow {\mathbb Z}_n$ resembles a
pyramid.

\annotation{Here a nice figure of a pyramidal tour and a pyramid shall be.}
\begin{figure}[htbp]
\centering%
\begin{mplibcode}
  beginfig(0)
    pair p[];
    p1 = (0,0); p2 = (100,0); p3 = (200,0); p4 = (300,0); p5 = (400,0);
    pair c[];
    c1 = (50,20); c2 = (210,30); c3 = (250,-10); c4 = (100,-20);
    path a;
    a = p1 .. controls c1 ..
        p2 .. controls c2 ..
        p4 ..
        p5 .. controls c3 ..
        p3 ..  controls c4 ..
        cycle ;
    for i = 1 upto length a:
      dotlabel.top(decimal i, p[i]);
      drawarrow subpath (i-1,i) of a withcolor 0.2;
    endfor
  endfig;
\end{mplibcode}
\ 
\caption{A pyramidal tour over $K_5$.}
\end{figure}

\subsubsection{Pyramidal tours and polynomial time}

It turns out that for this particular exponential neighbourhood a simple dynamic programming
scheme delivers a polynomial solution.
\annotation{Is there any difference at all between sTSP and aTSP here?}
(For simplicity, we shall consider symmetric TSP (sTSP).)
For a sTSP of size $n$, let
\begin{align*}
  \Phi\left(i,j\right) \ldots \; & \text{shortest pyramidal path} \left(i,k_1,\ldots,k_p,j\right) \\
                      & \text{s.t.} \; \left\{ k_1,\ldots,k_p \right\} = \left\{ k,\ldots,n \right\} \\
                      & \text{where} \; k = \max\left\{i,j\right\}+1.
\end{align*}
One can easily see that $\Phi(1,1)$ is the optimum of all pyramidal tours in $S_n$ (one-based).
\[
  \Phi\left(i,n\right) = \left(i,n\right) \; \text{and} \; \Phi\left(n,j\right) = \left(n,j\right).
\]
\annotation{A construction illustration would be in order here.}
Now note that one can build an arbitrary pyramidal path by considering nodes' positions in increasing order:
given a partial tour (a pyramidal path) $\left(1,\ldots,i,j,\ldots,1\right)$ which increases to $i$ and
decreases from $j$, the next element ($\max\left\{i,j\right\}+1$) has to lie adjacent to $i$ or $j$.
Since every pyramidal tour over $K_n$ can be constructed in this manner, it follows that
\begin{gather*}
  \Phi\left(i,j\right) = \argmin_{\tau \in T}\mathfrak{w}\left(\tau\right) \\
  \text{where} \; T = \left\{ \left( i,k \right) \oplus \Phi(k,j), \Phi(i,k) \oplus \left( k,j \right) \right\} \\
  \text{and} \; \oplus \; \ldots \; \text{path concatenation}.
\end{gather*}
We hence arrive at
\begin{theorem}
  For $K_n$, amongst $2^{n-3}$ pyramidal tours an optimum can be found in $\mathcal O(n^2)$ steps.
\end{theorem}
\begin{proof}
  For $K_n$, computing the optimum now requires computing not more than $\Phi\left(i,j\right)$
  for all pairs $\left(i,j\right)$.
\end{proof}

\begin{figure}[htbp]
\centering
\includegraphics[width=0.59\textwidth]{../plot/out/trace_pyr.pdf}
\caption{A trace of our implementation's pyramidal search graph traversal over $K_5$.}
\end{figure}

\subsubsection{Local search heuristics}

\annotation{Demidenko conditions: is a description necessary/interesting?}
While determining if a given TSP instance is pyramidal is NP-hard \cite{baki},
a number of polynomially testable classes which are pyramidally solvable exist \cite{baki,gilmore} ---
still, those are not widely common.

Pyramidal search can also be extended to a local search heuristic:
note first how the tour obtained depends on node numbering.
Now let $\operatorname{pyr}(C)$ denote best pyramidal tour given a distance matrix $C$
and, given a tour $\tau$, $\sigma_\tau \left(C\right)$ be the permutation associated with $\tau$
applied to rows and columns of $C$ (i.e. renumbering of nodes).
Let
\[
  h_C \; : \;
    \tau \mapsto \left(
      \sigma_{\tau}^{-1} \circ
      \operatorname{pyr} \circ
      \sigma_{\tau}
    \right)
    \left( C \right).
\]
Then $h_C\left(\left(1,2,\ldots,n,1\right)\right) = \operatorname{pyr}\left(C\right)$.
Since $\left(1,2,\ldots,n,1\right)$ is a pyramidal tour,
\[
  \operatorname{len} \circ h_C \leq \operatorname{len}
\]
holds everywhere.
In this way, iterating from $\left(1,2,\ldots,n,1\right)$ to a fixed point of $h_C$ yields a
local search heuristic.

It had been shown \cite{belperm} that we can significantly improve on this by enhancing
$\operatorname{pyr}$ to choose from $n$ shortest pyramidal tours obtained by applying
cyclic permutations to initial node ordering.

\subsection{Four point conditions exponential neighbourhoods}

\subsection{Strongly balanced tours}

\annotation{Balanced tours interesting?}
Another class of sTSP is the {\em Relaxed Supnick TSP} \/(RS-TSP), which is
obtained by restricting considered instances via a simple four-point
condition:
\[
  c_{ik} + c_{jl} \leq c_{il} + c_{jk} \quad \forall \; 1 \leq i < j < k < l \leq n,
\]
where $\left(c_{ij}\right)$ is the distance matrix.

While it has been shown that RS-TSP is NP-hard, an exponential subset of
said problem which is polynomially solvable has also been found.  Herein we
present first implementation of the solution, which bears distinct similarity
to pyramidal TSP.

\annotation{Add supplementary pictures/explanation.}
We shall define our neighbourhood through explicit tour construction.  For brevity,
let $\left[a,b\right]$ describe a (simple) path $\left<a,\ldots,b\right>$.
\begin{define}[Strongly Balanced Tour]
  Consider following family of path sets:
  \begin{itemize}
    \item $\mathcal T_1 = \left\{\left<1\right>\right\};$
    \item $\mathcal T_n = \bigcup_{\tau \in \mathcal T_{n-1}}
      \left(
      \left\{
          \operatorname{add}\left(\tau,n\right),
          \operatorname{app}\left(\tau,n\right)
      \right\} \cup
      \operatorname{merge}\left(\tau,n\right)
      \right)
    $
    where
    \begin{align*}
      \operatorname{add}\left(\tau,n\right) = & \; \tau \cup \left\{\left<n\right>\right\}; \\
      \operatorname{app}\left(\tau,n\right) = & \;
        \tau \setminus \left\{\left[a_1,a_2\right]\right\}
        \cup \left\{\left<n\right>\oplus\left[a_1,a_2\right]\right\} \\
        & \; \text{for} \; \tau = \left\{ \left[a_i,a_{i+1}\right] \right\}_{i \in I}
        \text{and} \; a_1 = \min_{k \in I \cup 1+I}\left\{a_k\right\}, \\
      \operatorname{merge}\left(\tau,n\right) = & \;
        \left\{
          \tau \setminus \left\{\left[a_1,a_2\right],\left[a_3,a_4\right]\right\}
          \cup \left\{\left[a_2,a_1\right]\oplus\left<n\right>\oplus\left[a_3,a_4\right]\right\}
        \right\} \\
        & \; \text{for} \; \tau, a_1 \; \text{as above,}
          \; a_3 = \min_{\substack{k \in I \cup 1+I\\\ k \neq 1, k \neq 2}}\left\{a_k\right\}.
    \end{align*}
    Here multiple appearances of $[.,.]$ notation with same indexes refer to same paths.
    Also note how $\operatorname{merge}\left(\tau,n\right)$ is empty if there is no $a_3$
    satisfying the constraints.

  \end{itemize}
  An element of $\mathcal T_n$ is said to be {\bfseries strongly balanced}.
  A tour $\left<k_1,\ldots,k_n,k_1\right>$ is said to be {\bfseries strongly balanced} if
  $\left<k_1,\ldots,k_n\right> \in {\mathcal T}_n$.
\end{define}

\subsubsection{Strongly balanced tours and linear time}

\annotation{Interesting?}
It can subsequently be shown that the set of strongly balanced tours contains optimal solution
to TSP restricted to Relaxed Supnick matrices which additionaly satisfy
\[
  c_{il} + c_{jk} + c_{km} \geq c_{ik} + c_{jm} + c_{kl}
  \quad \forall i < j < k < l < m
\]
and that this condition can be checked in quadratic time.

\annotation{Details?}
If we further restrict this set by
\[
  c_{ji} + c_{kl} \leq c_{jl} + c_{ki} \quad \forall i<j, \; j+h \leq k < l
\]
for a given constant $h$, then we can show that the optimum can be found
amongst balanced tours for which number of paths of each element of $\mathcal T$
does not exceed $h/2 + 1$, and for such families size of ${\mathcal T}_n$ is bounded
by a constant for given $h$ \cite{fpneighb,fpclass}.

Number of unique node types, which presents upper bound for tree width, can
be seen in \autoref{tbl:nodeTypes}.  A breadth-first search implementation
could be used to naturally compute the tree width.

\annotation{Show how number of tours is $\Omega(2^n)$.}
Noting how the definition of strongly balanced tours naturally
yields a dynamic programming recursion, this results in a linear time
solution for the given exponential neighbourhood.

\setlength{\aboverulesep}{0pt}
\setlength{\belowrulesep}{0pt}
\renewcommand{\tabcolsep}{9pt}
\setlength{\extrarowheight}{1.1ex}
\newcolumntype{g}{>{\columncolor{Gray}}c}

\begin{table}[htpb]
\centering
    \mbox{%
      \pgfplotstabletypeset[
        col sep=space,
        std=-10:10,
        every head row/.style={output empty row},
        every head row/.style={%
          before row={\toprule},
          after row={\midrule}
        },
        display columns/0/.style={%
          column name={\bfseries Node size limit},
          column type={g}
        },
        display columns/1/.style={%
          column name={\bfseries Node types},
          column type={r}
        },
        display columns/2/.style={%
          column name={\bfseries Stable at depth},
          column type={g}
        },
        every last row/.style={after row=\bottomrule},
        %multicolumn names,
      ]{../plot/data/num\string_types.csv}%
    }
\caption{Number of node types encountered.  This seems to closely reflect the tree width -- and thus runtime -- growth.}%
\label{tbl:nodeTypes}%
\end{table}

\begin{figure}[htbp]
\makebox[\textwidth][c]{%
  \includegraphics[width=1.21\textwidth]{../plot/out/trace_bal.pdf}
}
\caption{%
  A trace of our implementation's strongly balanced search graph traversal over $K_8$.  Here, maximum node size is 2.
}
\end{figure}

\subsubsection{Local search heuristics}

Having constructed a linear solution to an exponential problem class, quite
naturally the question arises whether this would constitute a good heuristic
in general case.
\annotation{Check: $\mu$ prepended or appended?}
To try and answer this, we have taken the same path as in pyramidal case.
Note that to account for strongly balanced tours structure, we have to
\begin{itemize}
  \item replace $\sigma_\tau$ by $\sigma_\tau \circ \sigma_\mu$
        in fixed point iteration, where $\mu = \left<\ldots,5,3,1,2,4,\ldots\right> \in S_n$;
  \item also apply $\mu$ (we also added $\mu$ mirrored)
        before considering cyclic permutations while also considering the identity permutation.
\end{itemize}

Results of our experiments are presented in next section
(where {\em adaptive search} \/refers to number of considered permutations growing at stale iterations
and {\em flipflop search} \/refers to alterating between pyramidal and strongly balanced search in each iteration).

\clearpage
\dictum[Donald Knuth]{You’re bound to be unhappy if you optimize everything.}
%\dictum[Edward V Berard]{Walking on water and developing software from a specification are easy if both are frozen.}
\section{Implementation}

\subsection{Brief overview}
\annotation{TODO}

Written in {\ttfamily Standard ML}.
Usable via SML REPL, standalone executable or shared library.
Tested under multiple software (see \autoref{app:compiler}) and OS/hardware platforms --
Linux (amd64) and Windows (x86) (MinGW-crosscompiled).

\annotation{Tradeoff illustration: provide LOC count?}
General dynamic programming results implementation via search graph traversal (simple depth-first search),
making a tradeoff between performance and extendibility.
BFS could be somewhat faster, but for our research dfs is enough.
Memoization via hash tables.

\annotation{Can we rename repo?}
This project's sources are available
\href{https://bitbucket.org/mad_hatter/rstsp/}{online}\footnote{\url{https://bitbucket.org/mad_hatter/rstsp/}}.

\subsection{Runtime behaviour}
\annotation{TODO}
\annotation{Describe \autoref{fig:benchRAND}.}

Clearly pronounced linear runtime for str.~balanced search, apparently heavy overhead penalty.
Pyramidal faster but str.~balanced eventually takes over.
See \autoref{fig:benchRAND}.
Memory usage not seriously measured, since bottleneck here is runtime.

\begin{figure}[htbp]
\null\hfill%
\subfloat[higher node sizes]{%
  \includegraphics[width=0.49\textwidth]{../plot/out/mlton_time_random_low.pdf}
}\hfill\null
\null\hfill%
\subfloat[medium node sizes]{%
  \includegraphics[width=0.49\textwidth]{../plot/out/mlton_time_random_fast.pdf}
}\hfill\null\\
\null\hfill%
\subfloat[low node sizes]{%
  \includegraphics[width=0.59\textwidth]{../plot/out/mlton_time_random_steady.pdf}
}\hfill\null\\
\null\hfill%
\subfloat[pyramidal vs str.~balanced/m3]{%
  \includegraphics[width=0.49\textwidth]{../plot/out/mlton_time_random_med.pdf}
}\hfill\null
\null\hfill%
\subfloat[pyramidal vs str.~balanced/m4]{%
  \includegraphics[width=0.49\textwidth]{../plot/out/mlton_time_random_hi.pdf}
}
\hfill\null
\caption{Single traversal running times (Linux 3.10.17 ({\ttfamily x86\_64}) @ i5-2520M).}
\label{fig:benchRAND}
\end{figure}

\subsection{Tour benchmarks}

\annotation{TODO}
Describe results, add Relaxed Supnick matrices benchmarks (!),
prettier (also, split) histograms?, plot iterations behavior? -- see \autoref{fig:benchTSPLIB}.

In what limited testing we have done, pyramidal still performs better.

For str.~balanced: adaptive search seems to perform better
than always choosing from $2n+1$ specific permutations.

\begin{figure}[htbp]
\null\hfill%
\subfloat[random --- tour length]{%
  \includegraphics[width=0.69\textwidth]{../plot/out/random_val.pdf}
}\hfill\null\\
\null\hfill%
\subfloat[tsplib \& vlsi --- tour quality]{%
  \includegraphics[width=0.69\textwidth]{../plot/out/tsplib_val.pdf}
}\hfill\null\\
\null\hfill%
\subfloat[tsplib \& vlsi --- time needed]{%
  \includegraphics[width=0.69\textwidth]{../plot/out/tsplib_time.pdf}
}\hfill\null
\caption{Benchmarks --- small instances.}
\label{fig:benchTSPLIB}
\end{figure}

\clearpage
\dictum[Arthur Bloch]{Enough research will tend to support your conclusions.}
\section{Conclusions}

\subsection{Results obtained}
\annotation{TODO}
First implementation of str.~balanced search.
Some benchmarks.
Preprocessing offers no gain.

\begin{figure}[htbp]
%\begin{wrapfigure}{r}{0.5\textwidth}
\centering
\includegraphics[width=0.54\textwidth]{./tsp.png}
\caption*{\captionsize Source: \small \url{https://xkcd.com/399/}.}
%\end{wrapfigure}
\end{figure}

\subsection{Areas for further research}
\annotation{TODO}

Are these heuristics better for aTSP\@?
Untested: other metrics or higher dimensions, dedicated implementations' performance.
Further heuristics combinations.
Branch cutting.
Different permutations set for str.~balanced tours?

\subsection{Dynamic programming algorithms and concurrency}
\annotation{TODO}

Dedicated implementations (pyramidal: bottom up easy) fast, but not always feasible (str.~balanced: machine-generated).
Naive concurrency (see the {\ttfamily experimental} code) not easily applicable in general case (dense search tree).
Potentially fruitful approaches: tuple spaces, message passing languages, etc
(PolyML/Linda?, manticore, jocaml, mythryl, aliceml, clojure/clotilde, scala/akka, $\ldots$).

%\input{problem1.tex}

%%
%% BS
%%
%Let
%\[ \Phi_k\left(\big\{ \{a_1,b_1\}, \ldots ,\{a_p,b_p\} \big\} \right) \]
%denote shortest (in total length) balanced paths set containing points $1 \ldots k$
%with $a_i, b_i$ being the paths' endpoints
%and
%\[ B\left(\big\{ \{a_1,b_1\}, \ldots ,\{a_p,b_p\} \big\}\right) = \bigcup_i\{a_i,b_i\}.\]
%%Then following holds:
%\begin{enumerate}
%  \item
%\[
%  \Phi_k\left(\big\{ \{a_1\}, \ldots , \{a_k\} \big\} \right) =
%  \left\{ \left< a_1 \right> , \ldots , \left< a_k \right> \right\}
%\]
%  \item
%\[
%  \Phi_k\left(\big\{ s_1, \ldots , s_p, \{k\} \big\} \right) =
%  \Phi_{k-1}\left(\big\{ s_1, \ldots , s_p \big\}\right) \cup
%  \left\{\left<k\right>\right\}
%\]
%  \item
%\[
%  \Phi_k\left(\big\{ s_1, \ldots , s_p, \{r,k\} \big\} \right) =
%  append\left(\Phi^{*},z^{\ast},k\right)
%\]
%where
%\[
%  \Phi^{*} = \Phi_{k-1}\left(\big\{ s_1, \ldots , s_p, \{r,z^{*}\} \big\} \right)
%\]
%\[
%  z^{*} = \argmin_{z \in [1,\ldots,m]}{
%    \left(
%      length \;
%      \Phi_{k-1}\left(\big\{ s_1, \ldots , s_p, \{r,z\} \big\} \right)
%      + d(z,k)
%    \right)
%  }
%\]
%\[
%  m = \min B\left(\{s_1, \ldots ,s_p\}\right) \cup \{k\} - 1
%\]
%  \item
%\[
%  \Phi_k\left(\left\{ s_1, \ldots , s_p \right\} \right) =
%  \;\text{with}\;
%  k \notin B\left({s_1,\ldots,s_p}\right)
%\]
%is then chosen from two possible solutions:
%\begin{enumerate}
%  \item
%\[
%  merge\left(\Phi_{q^{*}},z^{*},z^{**},k\right)
%\]
%where
%\[
%  \Phi_{q^{*}} = \Phi_{k-1}\left(\big\{ s_1, \ldots , s_{q^{*}-1},
%  \{z^{*},a^{*}\}, \{b^{*},z^{**}\}, s_{q^{*}+1}, \ldots, s_p \big\} \right)
%\]
%\[
%  s_{q^{*}} = \{a^{*},b^{*}\}
%\]
%\[
%  (q^{*}, z^{*},z^{**}) = \argmin_{
%    \substack{q \in [1,p] \\\
%              z^\prime \in [1,\ldots,m-1] \\\
%              z^{\prime\prime} \in [z^\prime+1, m-1]}}{
%    \mkern-18mu \Phi_{k-1}\left(\big\{ s_1, \ldots , s_{q-1},
%    \{z^{\prime},a\}, \{b,z^{\prime\prime}\},
%    s_{q+1}, \ldots ,s_p \big\} \right)
%    \! + \! d(z^\prime,a) \! + \! d(z^{\prime\prime},b)
%  }
%\]
%\[
%  m = \min B\left(\{s_1,\ldots,s_p\}\right)
%\]
%  \item
%\[
%  merge\left(\Phi^{*},z^{*},z^{**},k\right)
%\]
%for
%\[
%  s_p = \{m,r\}, m = \min B\left(\{s_1,\ldots,s_p\}\right), m \neq r
%\]
%where
%\[
%  \Phi^{*} = \Phi_{k-1}\left(\big\{ s_1, \ldots , s_{p-1}, \{z^{*},m\}, \{r,z^{**}\} \big\} \right)
%\]
%\[
%  (z^{*},z^{**}) = \argmin_{\substack{z^\prime \in [1,\ldots,m-1] \\\
%  z^{\prime\prime} \in [m+1, \min B\left(\{s_1,\ldots,s_{p-1}\}\right)-1]}}{
%    \mkern-28mu \mkern-18mu
%    \Phi_{k-1}\left(\big\{ s_1, \ldots , s_{p-1},
%    \{z^{\prime},m\}, \{r,z^{\prime\prime}\} \big\} \right)
%    + d(z^\prime,m) + d(z^{\prime\prime},r)
%  }
%\]
%\end{enumerate}
%\end{enumerate}
%
%\cite{example}
%

%\appendix
%\addcontentsline{toc}{chapter}{Appendices}
%\makeatletter
%\addtocontents{toc}{\let\protect\l@chapter\protect\l@section}
%\makeatother
%\chapter{First Appendix}
%\chapter{Second Appendix}

\clearpage
\begin{appendices}

\makeatletter
\addtocontents{toc}{\let\protect\l@chapter\protect\l@section}
\makeatother

%\section{Code Index}
%\lstlistoflistings%

\section{Recursion and Time Complexity}

\section{Sample SML REPL Session}

\section{Compiler Support}
\label{app:compiler}
\input{sml-support.tex}

\section{Shared Library Interface}
\label{app:so}

\needspace{5\baselineskip}
\mylisting[
  caption={[FFI usage example]FFI usage example.},
  firstline=8,fontadjust,language=C,
  label=lst:so]{../src/rstsp/librstsp/test.c}

%%
% $Id$
% $Author$
% $Date$
% $Rev: 34 $
%

\needspace{5\baselineskip}

\mylisting[
  caption={[Test Code]
  Test.  Code.},
  linerange={56-99},fontadjust,language=ML,
  label=lst:pyrTest]{../src/rstsp/tspgraph/pyrgraph.sml}



\end{appendices}

%\printindex
\printbibliography%

\end{document}

